% !TeX root = ./bnuthesis-example.tex

% 论文基本信息配置

\bnusetup{
  %******************************
  % 注意:
  %   1. 配置里面不要出现空行
  %   2. 不需要的配置信息可以删除
  %   3. 建议先阅读文档中所有关于选项的说明
  %******************************
  %
  % 输出格式
  %   选择打印版(print)或用于提交的电子版(electronic),前者会插入空白页以便直接双面打印
  %
  output = print,
  %
  % 标题
  %   可使用“\\”命令手动控制换行
  %
  title  = {北京师范大学学位论文 \LaTeX{} 模板使用示例文档 v\version},
  title* = {AN INTRODUCTION TO \LaTeX{} THESIS TEMPLATE OF BEIJING NORMAL UNIVERSITY v\version},
  %
  % 学位
  %   1. 学术型
  %      - 中文
  %        需注明所属的学科门类,例如:
  %        哲学、经济学、法学、教育学、文学、历史学、理学、工学、农学、医学、
  %        军事学、管理学、艺术学
  %      - 英文
  %        博士:Doctor of Philosophy
  %        硕士:
  %          哲学、文学、历史学、法学、教育学、艺术学门类,公共管理学科
  %          填写“Master of Arts“,其它填写“Master of Science”
  %   2. 专业型
  %      直接填写专业学位的名称,例如:
  %      教育博士、工程硕士等
  %      Doctor of Education, Master of Engineering
  %   3. 本科生不需要填写
  %
  degree-name  = {理学博士},
  degree-name* = {Doctor of Science},
  %
  % 培养单位
  %   填写所属院系的全名
  %
  department = {数学科学学院},
  %
  % 学科
  %   1. 学术型学位
  %      获得一级学科授权的学科填写一级学科名称,其他填写二级学科名称
  %   2. 工程硕士
  %      工程领域名称
  %   3. 其他专业型学位
  %      不填写此项
  %   4. 本科生填写专业名称,第二学位论文需标注“(第二学位)”
  %
  discipline  = {计算数学},
  discipline* = {Mathematics and Applied Mathematics (Liyun)},
  %
  % 学号
  %
  student-id  = {202121130093},
  %
  % 姓名
  %
  author  = {阙嘉豪},
  author* = {Que Jiahao},
  %
  % 指导教师
  %   中文姓名和职称之间以英文逗号“,”分开,下同
  %
  supervisor  = {陈华杰, 教授},
  supervisor* = {Professor Chen Huajie},
  supervisor-department = {数学科学学院},
  %
  % 副指导教师
  %
  associate-supervisor  = {陈华杰, 教授},
  associate-supervisor* = {Professor Chen Huajie},
  %
  % 联合指导教师
  %
  % co-supervisor  = {某某某, 教授},
  % co-supervisor* = {Professor Mou Moumou},
  %
  % 日期
  %   使用 ISO 格式;默认为当前时间
  %
  % date = {2019-07-07},
  %
  % 是否在中文封面后的空白页生成书脊(默认 false)
  %
  include-spine = true,
  spine-author = {北京师范大学},
  %
  % 密级和年限
  %   秘密, 机密, 绝密
  %
  % secret-level = {秘密},
  % secret-year  = {10},
  %
  % 博士后专有部分
  %
  % clc                = {分类号},
  % udc                = {UDC},
  % id                 = {编号},
  % discipline-level-1 = {计算机科学与技术},  % 流动站(一级学科)名称
  % discipline-level-2 = {系统结构},          % 专业(二级学科)名称
  % start-date         = {2011-07-01},        % 研究工作起始时间
}

% 载入所需的宏包

% 定理类环境宏包
\usepackage{amsthm}
% 也可以使用 ntheorem
% \usepackage[amsmath,thmmarks,hyperref]{ntheorem}

\bnusetup{
  %
  % 数学字体
  % math-style = GB,  % GB | ISO | TeX
  math-font  = xits,  % sitx | xits | libertinus
}

% 可以使用 nomencl 生成符号和缩略语说明
% \usepackage{nomencl}
% \makenomenclature

% 表格加脚注
\usepackage{threeparttable}

% 表格中支持跨行
\usepackage{multirow}

% 固定宽度的表格。
% \usepackage{tabularx}

% 跨页表格
\usepackage{longtable}

% 量和单位
\usepackage{siunitx}

% 参考文献使用 BibTeX + natbib 宏包
% 顺序编码制
\usepackage[sort]{natbib}
\bibliographystyle{bnuthesis-numeric}

% 著者-出版年制
% \usepackage{natbib}
% \bibliographystyle{bnuthesis-author-year}

% 本科生参考文献的著录格式
% \usepackage[sort]{natbib}
% \bibliographystyle{bnuthesis-bachelor}

% 参考文献使用 BibLaTeX 宏包
% \usepackage[backend=biber,style=bnuthesis-numeric]{biblatex}
% \usepackage[backend=biber,style=bnuthesis-author-year]{biblatex}
% \usepackage[backend=biber,style=apa]{biblatex}
% \usepackage[backend=biber,style=mla-new]{biblatex}
% 声明 BibLaTeX 的数据库
% \addbibresource{ref/refs.bib}

% 定义所有的图片文件在 figures 子目录下
\graphicspath{{figures/}}

% 数学命令
\makeatletter
\newcommand\dif{%  % 微分符号
  \mathop{}\!%
  \ifbnu@math@style@TeX
    d%
  \else
    \mathrm{d}%
  \fi
}
\makeatother

% 算法环境
\usepackage{algorithm}
\usepackage{algpseudocode}
\floatname{algorithm}{\mdseries\heiti 算法}
\renewcommand{\algorithmicrequire}{\heiti 输入:}
\renewcommand{\algorithmicensure}{\heiti 输出:}
\DeclareCaptionFont{song}{\songti}
\DeclareCaptionFont{minusfive}{\zihao{-5}}
\captionsetup*[algorithm]{	% 算法标题设置
  font={song, minusfive}	% 宋体小五
}
\AtBeginEnvironment{algorithmic}{%
  \zihao{-5}%
}

% 使用 pgf 图片
\usepackage{pgf}

% 注释环境
\usepackage{comment}

% hyperref 宏包在最后调用
\usepackage{hyperref}
