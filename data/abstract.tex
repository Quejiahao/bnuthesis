% !TeX root = ../bnuthesis-example.tex

% 中英文摘要和关键字

\begin{abstract}
  本科生。

  摘要是对论文内容不加注释和评论的简明归纳,应包括研究工作的目的、方法和结论,重点是结果和结论。用语要规范,一般不用公式和非规范符号术语,不出现图、表、化学结构式等。采用第三人称撰写,一般在 300 字左右。

  论文应附有英文题目和英文摘要以便于进行国际交流。英文题目和英文摘要应明确、简练,其内容包括研究目的、方法、主要结果和结论。一般不宜超过 250 个实词。

  关键词是为了满足文献标引或检索工作的需要而从论文中选取出的用以表示全文主题内容信息的词或词组。关键词包括主题和自由词:主题词是专门为文献的标引或检索而从自然语言的主要词汇中挑选出来并加以规范化了的词或词组;自由词则是未规范化的即还未收入主题词表中的词或词组。

  每篇论文中应列出 $3 \sim 8$ 个关键词,它们应能反映论文的主题内容。其中主题词应尽可能多一些,关键词作为论文的一个组成部分,列于摘要段之后。还应列出与中文对应的英文关键词(Keywords)。关键词间空格 1 个字符。

  研究生。

  (1)摘要是学位论文的内容不加注释和评论的简短陈述。一般另页置于题名页后。

  (2)学位论文应有摘要,为了国际交流,还应有外文(多用英文)摘要。

  (3)摘要应具有独立性和自含性,即不阅读论文的全文,就能获得必要的信息。摘要中有数据、有结论,是一篇完整的短文,可以独立使用,可以引用。
  摘要的内容应包含与报告、论文等同量的主要信息,供读者确定有无必要阅读全文,也可供二次文献(文摘等)采用。摘要一般应说明研究工作的目的、实验方
  法、结果和最终结论等。重点是结果和结论。注意突出具有创新性的成果和新见解。

  (4)中文摘要一般 1000 字左右,外文摘要 1-2 页。如遇特殊需要字数可以略多。

  (5)除了无法变通的办法可用以外,摘要中不用图、表、化学结构式、非公知公用的符号和术语。

  关键词是为了文献标引而从学术论文中选取出来用以表示全文主题内容信息款目的单词或术语。

  每篇论文选取 3-8 个关键词,用显著的字符另起一行,排在摘要的左下方。如有可能,尽量采用《汉语主题词表》等词表提供的规范词。

  % 关键词用“英文逗号”分隔,输出时会自动处理为正确的分隔符
  \bnusetup{
    keywords = {关键词 1, 关键词 2, 关键词 3, 关键词 4, 关键词 5},
  }
\end{abstract}

\begin{abstract*}
  An abstract of a dissertation is a summary and extraction of research work and contributions.
  Included in an abstract should be description of research topic and research objective, brief introduction to methodology and research process, and summarization of conclusion and contributions of the research.
  An abstract should be characterized by independence and clarity and carry identical information with the dissertation.
  It should be such that the general idea and major contributions of the dissertation are conveyed without reading the dissertation.

  An abstract should be concise and to the point.
  It is a misunderstanding to make an abstract an outline of the dissertation and words “the first chapter”, “the second chapter” and the like should be avoided in the abstract.

  Keywords are terms used in a dissertation for indexing, reflecting core information of the dissertation.
  An abstract may contain a maximum of 5 keywords, with semi-colons used in between to separate one another.

  % Use comma as seperator when inputting
  \bnusetup{
    keywords* = {Keyword 1, Keyword 2, Keyword 3, Keyword 4, Keyword 5},
  }
\end{abstract*}
